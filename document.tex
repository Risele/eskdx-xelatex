% !TeX program = xelatex

%%% Загружаем заголовочный файл, который хранит все настройки и все
%%% подгружаемые пакеты
\newcommand{\No}{\textnumero}

\input{input/header.tex}

%%% Загружаем настройки пакета eskdx, там нужно заполнить информацию
%%% о документе - ФИО авторов, название документов и т.п.
\input{input/eskdx.tex}

\begin{document}

%%% Делаем титульник
\maketitle

%%% Делаем содержание
\tableofcontents

%%% Введение пишется без цифры и добавляется в содержание
\section*{Введение}
\addcontentsline{toc}{section}{Введение}
Здесь текст введения.

Душа моя озарена неземной радостью, как эти чудесные весенние утра, которыми я наслаждаюсь от всего сердца. Я совсем один и блаженствую в здешнем краю, словно созданном для таких, как я. Я так счастлив, мой друг, так упоен ощущением покоя, что искусство мое страдает от этого. Ни одного штриха не мог бы я сделать, а никогда не был таким большим художником, как в эти минуты. Когда от милой моей долины поднимается пар и полдневное солнце стоит над непроницаемой чащей темного леса и лишь редкий луч проскальзывает в его святая святых, а я лежу в высокой траве у быстрого ручья и, прильнув к земле, вижу тысячи всевозможных былинок и чувствую, как близок моему сердцу крошечный мирок, что снует между стебельками, наблюдаю эти неисчислимые, непостижимые разновидности червяков и мошек и чувствую близость всемогущего, создавшего нас по своему подобию, веяние вселюбящего, судившего нам парить в вечном блаженстве, когда взор мой туманится и все вокруг меня и небо надо мной запечатлены в моей душе, точно образ возлюбленной, - тогда, дорогой друг, меня часто томит мысль: "Ах! Как бы выразить, как бы вдохнуть в рисунок то, что так полно, так трепетно живет во мне, запечатлеть отражение моей души, как душа моя - отражение предвечного бога!" Друг



\section{Заголовок}
Здесь основной текст работы\cite{BookRef, ArticleRef, LinkRef}.

%%% Заключение, так же как и введение выводим без цифры, добавляем в содержание
\section*{Заключение}
\addcontentsline{toc}{section}{Заключение}
Здесь текст заключения

%%% Далее выводим библиографию
%%% Исправляем ошибку библиографии - «кавычка перед тире»
%%% http://ru-tex.livejournal.com/105178.html?thread=801498
\catcode`"\active\def"{\relax}
\bibliographystyle{gost780s}
\bibliography{bibliography}{}
%%% Если пропадают инициалы, смотрите сюда:
%%% http://plumbum-blog.blogspot.com/2010/10/bibtex-miktex-gost780s.html
\end{document}
