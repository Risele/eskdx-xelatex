% !TeX program = xelatex

%%% Загружаем заголовочный файл, который хранит все настройки и все
%%% подгружаемые пакеты
\newcommand{\No}{\textnumero}

\input{input/header.tex}

%%% Загружаем настройки пакета eskdx, там нужно заполнить информацию
%%% о документе - ФИО авторов, название документов и т.п.
\input{input/eskdx.tex}

\begin{document}

%%% Делаем титульник
\maketitle

%%% Делаем содержание
\tableofcontents

%%% Введение пишется без цифры и добавляется в содержание
\section*{Введение}
\addcontentsline{toc}{section}{Введение}
Здесь текст введения.

Проснувшись <<однажды утром>> после беспокойного сна, Грегор Замза обнаружил, что он у себя в постели превратился в страшное насекомое. 
Лежа на панцирнотвердой спине, он видел,

\begin{enumerate}
   \item стоило ему приподнять голову, 
   \item свой 
   \begin{enumerate}
      \item коричневый, 
      \item выпуклый, 
      \item разделенный дугообразными чешуйками живот, 
   \end{enumerate}
   
   на верхушке которого еле держалось готовое вот-вот окончательно сползти одеяло. 
   
\end{enumerate}

Его многочисленные, убого тонкие по сравнению с остальным телом ножки беспомощно копошились у него перед глазами. «Что со мной случилось?» – подумал он. Это не было сном. Его комната, настоящая, разве что слишком маленькая, но обычная комната, мирно покоилась в своих четырех хорошо знакомых стенах. Над столом, где были разложены распакованные образцы сукон – Замза был коммивояжером, – висел портрет, который он недавно вырезал из иллюстрированного журнала и вставил в красивую золоченую рамку. На портрете была изображена дама в меховой шляпе и боа, она сидела очень прямо и протягивала зрителю тяжелую меховую муфту, в которой целиком исчезала ее рука. Затем взгляд Грегора устремился в окно, и пасмурная погода – слышно было, как по жести подоконника стучат капли дождя – привела его и вовсе в грустное настроение. «Хорошо бы еще немного поспать и забыть всю эту чепуху», – подумал он, но это было совершенно неосуществимо, он привык спать на правом боку, а в теперешнем своем





\section{Заголовок}
Здесь основной текст работы\cite{BookRef, ArticleRef, LinkRef}.

%%% Заключение, так же как и введение выводим без цифры, добавляем в содержание
\section*{Заключение}
\addcontentsline{toc}{section}{Заключение}
Здесь текст заключения

%%% Далее выводим библиографию
%%% Исправляем ошибку библиографии - «кавычка перед тире»
%%% http://ru-tex.livejournal.com/105178.html?thread=801498
\catcode`"\active\def"{\relax}
\bibliographystyle{gost780s}
\bibliography{bibliography}{}
%%% Если пропадают инициалы, смотрите сюда:
%%% http://plumbum-blog.blogspot.com/2010/10/bibtex-miktex-gost780s.html
\end{document}
