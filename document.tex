% !TeX program = xelatex

%%% Загружаем заголовочный файл, который хранит все настройки и все
%%% подгружаемые пакеты
\newcommand{\No}{\textnumero}

\input{input/header.tex}

%%% Загружаем настройки пакета eskdx, там нужно заполнить информацию
%%% о документе - ФИО авторов, название документов и т.п.
\input{input/eskdx.tex}

\begin{document}

%%% Делаем титульник
\maketitle

%%% Делаем содержание
\tableofcontents

%%% Введение пишется без цифры и добавляется в содержание
\section*{Введение}
\addcontentsline{toc}{section}{Введение}
Здесь текст введения.

\section{Заголовок}
Здесь основной текст работы\cite{BookRef, ArticleRef, LinkRef}.

%%% Заключение, так же как и введение выводим без цифры, добавляем в содержание
\section*{Заключение}
\addcontentsline{toc}{section}{Заключение}
Здесь текст заключения
hjfолпро
пароаор
ромапроэ
прмро
ормрр
ирпрропапра
орпароапп
олпррополпр
пррпрроп
лорпроппр
орполпрол

Интернет-канал Анна-Ньюс продолжает цикл передач "Без цензуры". И сегодня в гостях у ведущего студии кандидата социологических наук, политолога Дмитрия Соина - член совета директоров НПО "Союз", заместитель начальника организационного отдела ЦК профсоюзов России по развитию регионов Сергей Георгиевич Романец. Темой сегодняшней беседы будет проблематика регионов России
%%% Далее выводим библиографию
%%% Исправляем ошибку библиографии - «кавычка перед тире»
%%% http://ru-tex.livejournal.com/105178.html?thread=801498
\catcode`"\active\def"{\relax}
\bibliographystyle{gost780s}
\bibliography{bibliography}{}
%%% Если пропадают инициалы, смотрите сюда:
%%% http://plumbum-blog.blogspot.com/2010/10/bibtex-miktex-gost780s.html
\end{document}
